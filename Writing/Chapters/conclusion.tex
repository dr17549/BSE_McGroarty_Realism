\section{Project Summary}

This initial goal of the project is introduce realistic trading dynamics to the BSE. To a certain extent, the goal has been achieved since we have successfully shown a market similar to what is described in Oesch \cite{Oesch} and Minor Price Spikes which exhibits the behaviour of Mean reversion and Momentum traders. In addition, we have successfully implemented a new BSE which can handle more complex order types and shown the system integrity through the Base Line results of the three agents : ZI-P, ZI-C and Klapan's Sniper. In addition, we have shown that the behaviour of the five agent listed in McG (Market maker, Liquidity consumer, Mean reversion trader, Momentum trader and Noise trader) is implemented as to what is described by both McG and Oesch as well as adapting their parameters so that the behaviours are more suited for the current implementation of the BSE. 

\section{Challenges}
Two of the most important challenges of this project is the time constraint and hardware challenges. Because of COVID-19, we were only able run the experiments on a personal computer, which might not as be as powerful as the ones available at the university, limiting the experiment trials and total run-time of the McG. 

In addition, because many parameters of McG and Oesch was not given, this poses a huge challenge in the implementation stages of the trading agents and the market. The parameters of the agent, as shown in various sections of this paper, plays a huge role in determining the behaviour of an agent as well as the market as a whole. Many of the parameters of the agents and market, such as the number of agents and their ratio, has been chosen through trial and error as well as educated guess. Because all the agent behaviour have been shown to act as expected, it is a matter of how the market is setup and their parameters that determine the return statistics and how close it is to McG and Oesch's market. 

\section{Future work}
Apart of mimicking the markets shown in Oesch and McG there are also other areas which are interesting to explore. In the background chapter, we mentioned that the dominance of agents such as AA and ZI-P can certainly change with the market it is operating in. Because the markets in Oesch and McG has properties that real market has, by experimenting with AA and ZI-P in this market may give new insights on how the agents will react in the real market and how certain aspects of a real market dynamics may change the behaviour of the trading agents. 

In addition, it would also be interesting to do a more in-depth study of price spikes and what kind of scenarios would create these price spikes. As mentioned in the background, the research of this ultrafast extreme events are very relevant in this decade. This includes exploration on how each individual agents are contributing to the price spikes and how their parameters, as well as their ratios in the market, can further effect the occurrence of these events.  